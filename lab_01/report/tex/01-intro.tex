\anonsection{ВВЕДЕНИЕ}

Шифровальная машина <<Энигма>> --- портативная шифровальная машина, использовавшаяся для шифрования и расшифрования секретных сообщений (семество электромеханических роторных машин, применявшихся с 1920-х годов)~\cite{попов1история}, \cite{бабаш2015информационная}.

Ротор --- вращающийся диск, расположенный вдоль вала, на котором изображены символы алфавита по порядку \cite{шолин2018алгоритм}.
На роторе имелись электрические контакты в количестве, равном мощности алфавита.
При соприкосновении контакты соседних роторов замыкают электрическую цепь \cite{бабаш2015информационная}.

Входное колесо --- колесо, соединяющее коммутационную панель или клавиатуру с роторами \cite{бабаш2015информационная}.

Рефлектор --- деталь, соединяющая контакты последнего ротора попарно, коммутируя ток через роторы в обратном направлении \cite{бабаш2015информационная}.
Он обеспечивает гарантию того, что процесс расшифрования симметричен процессу шифрования, и свойство, заключающееся в том, что никакая буква не может быть зашифрована собой.

Цель данной лабораторной работы --- реализация в виде программы электронного аналога шифровальной машины «Энигма».

Для достижения поставленной цели требуется решить следующие задачи:
\begin{enumerate}
	\item описать алгоритм работы шифровальной машины <<Энигма>>;
	\item спроектировать описанный алгоритм;
	\item выбрать необходимые для разработки средства и разработать реализацию спроектированного алгоритма.
\end{enumerate}

Требования к выполнению лабораторной работы:
\begin{itemize}
	\item обеспечить шифрование и расшифровку произвольного файла, а также текстового сообщения с использованием разработанной программы;
	\item мощность шифруемого алфавита не должна превышать 64 символа;
	\item необходимо предусмотреть работу программы с пустым, однобайтовым файлом;
	\item должна быть возможность обработки файла архива (rar, zip или др.).
\end{itemize}
