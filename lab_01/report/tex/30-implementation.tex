\section{Технологическая часть}

Для выполнения реализации спроектированного алгоритма был выбран язык программирования C.

На листингах \ref{lst:msg}, \ref{lst:symbol} представлены реализации алгоритмов шифрования сообщения и символа соответственно.

\begin{listing}[H]
	\caption{Алгоритм шифрования сообщения}
	\inputminted[frame=single,fontsize = \footnotesize, linenos, breaklines, xleftmargin = 1.5em,breaksymbol = ""]{C++}{../lst/msg.cpp}
	\label{lst:msg}
\end{listing}

\begin{listing}[H]
	\caption{Алгоритм шифрования символа}
	\inputminted[frame=single,fontsize = \footnotesize, linenos, breaklines, xleftmargin = 1.5em,breaksymbol = ""]{C++}{../lst/symb.cpp}
	\label{lst:symbol}
\end{listing}

\subsection{Тестирование}

В таблице \ref{tab:test} приведены функциональные тесты (черный ящик).

Тесты пройдены успешно.

\begin{table}[htb]
	\caption{\centering Сравнение существующих решений}
	\small
	\centering\begin{tabular}{| c | c |}
		\hline
		Входная строка				& Выходная строка \\\hline
		WHATISDEADMAYNEVERDIE & IVXFMCXTOHYPGYWDYFGLB \\\hline
		IVXFMCXTOHYPGYWDYFGLB & WHATISDEADMAYNEVERDIE \\\hline
		<<>> & <<>> \\\hline
		A & L \\ \hline
		L & A \\ \hline
	\end{tabular}
	\label{tab:test}
\end{table}

Так же тесты проводились на файлах, пройдены успешно.

Содержимое исходного файла: lalala

Содержимое после шифрации зашифрованного файла: lalala.